\documentclass[letterpaper,11pt]{report}
\usepackage[T1]{fontenc}
\usepackage[utf8]{inputenc}
\usepackage{lmodern, amsmath, amsfonts, amssymb,amsthm,graphicx,color,xcolor,url,theorem,textcomp,glossaries,parskip,hyperref}

\title{Analyzing the Card Game BS}
\author{Iniyan Joseph}
\date{}

\begin{document}

\maketitle
\tableofcontents

\chapter{Rules of the Game}
\begin{description}
    \section{Objective of the Game}
    \item [Objective] In this game, each player ($i$) has the objective of disposing of all of their cards.
    \section{Setup of the Game}
    \item Let there be $n$ players, $3\leq n\leq 7, n\in\mathbb{N}$.
    \item Let there be a shared pot into which players may discard cards.
    \item Let there be a global type $\alpha$, defined as $\alpha_{0} = 1; \alpha_{t} = (\alpha_{t-1} \bmod 13) + 1$.
    \item The game uses a deck of 52 cards consisting of 4 suites. Each suite is $\{x | x \in \mathbb{N} \and x \in [1, 13] \}$.
    \item Initially, each player is given $\lfloor\frac{52}{n}\rfloor$, and $52 \bmod n$ cards are placed in the pot.
    \section{Defining a Player}
    \item Let the hand of $i$ be represented by a set $\{p_{i 1}, p_{i 2}, ..., p_{i 12}, p_{i 13}\}$ where $p_{i r}$ is the number of cards of type $r$, and $i$ is the player.
    \item Assume $\forall_{j}\sum_{i=1}^{n} p_ij = 4$
    \section{Defining the pot}
    \item Let the pot be represented as player $\gamma$
    \section{Defining a Turn}
    \item Play is conducted counter-clockwise, and no player may be skipped.
    \item In a turn, consider the players $i$ and $j, i \neq j$.
    \item Let a turn ($t$) consist of the following sequence:
    \begin{enumerate}
        \item Player $i$ discards $k_t$ cards; $k_t \in [1,4], k_t \in \mathbb{N}$
        \item Player $j$ may "call BS"
        \begin{enumerate}
            \item $\forall type(k_t) = \alpha_t \rightarrow$ j must take the pot
            \item $\exists type(k_t) \neq \alpha_t \rightarrow$ i must take the pot
        \end{enumerate}
    \end{enumerate}
    \section{The Problem}
    \paragraph{} Assume you are player 1 $p_{1 r}$. We must calculate $P(p_{i \alpha_t} \geq k_t)$
    \section{Levels of Thought}
    \begin{enumerate}
        \item The pot is empty and we have no knowledge of the player's cards.
        \item If a player $i$ tells the truth and player $j$ calls bluff in turn $t$, player $j$ definitely holds $k_t$ cards of type $\alpha_t$.
        \item The pot is treated as another player to whom cards are given in each turn
        \item The pot definitely holds the cards that you have placed into it in your previous turns, and when the pot is taken by a player $i$, $i$ definitely holds those cards.
    \end{enumerate}
\end{description}
\chapter{Describing the Initial State with $\forall_{i\in\gamma} i = 0$}
\section{Assumptions} 
\paragraph*{} To begin, let us find a formula to compute the relevant probability given the pot is empty and that there is no other information provided.
\paragraph*{Relevant Variables} The number of cards to distribute between $n-1$ players is $4-p_{0 \alpha_{t}}$. The number of cards player $i$ places is $k_t$, and the true amount they hold is $p_{i \alpha_t}$.
\begin{description}
    \item In general, to split $y$ items among $z$ people, There are ${y+z-1 \choose z-1}$ ways.
    \item We must count the number of ways in which $4-p_{0 \alpha_{t}}$ may be distributed between $n-1$ players. This is ${2-p_{0 \alpha_{t}} + n \choose n-2}$
    \item We must count the number of ways in which players $j$ such that $j\neq i$ can hold, which is ${1-p_{0 \alpha_{t}} - k_t + n \choose n-3}$ 
    \item We must count the number of ways in which player $i$ can hold $k_t$ cards, which is ${n-1 \choose k_t}$
    \linebreak
    \item This means that in the most naive case, the probability of player $i$ having p cards is  $\frac{{n-1 \choose k_t}\ast{1-p_{0 \alpha_{t}} - k_t + n \choose n-3}}{{2-p_{0 \alpha_{t}} + n \choose n-2}}$
\end{description}
\end{document}