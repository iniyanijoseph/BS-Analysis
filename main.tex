\documentclass[letterpaper,11pt]{report}
\usepackage[T1]{fontenc}
\usepackage[utf8]{inputenc}
\usepackage{lmodern, amsmath, amsfonts, amssymb,amsthm,graphicx,color,xcolor,url,theorem,textcomp,glossaries,parskip,hyperref}

\title{Analyzing the Card Game BS}
\author{Iniyan Joseph}
\date{}

\begin{document}

\maketitle
\tableofcontents

\chapter{Rules of the Game}
\begin{description}
    \section{Objective of the Game}
    \item [Objective] In this game, each player ($i$) has the objective of disposing of all of their cards.
    \section{Setup of the Game}
    \item Let there be $n$ players, $3\leq n\leq 7$.
    \item Let there be a shared table into which players may discard cards.
    \item Let there be a global number $\alpha$, defined as $\alpha_{0} = 1; \alpha_{t} = (\alpha_{t-1} \bmod 13) + 1$.
    \item The game uses a deck of 52 cards consisting of 4 suites, with exactly $13$ cards in each suite. 
    \item Initially, each player is given $\lfloor\frac{52}{n}\rfloor$, and the remaining $52 \bmod n$ cards are placed on the table.
    \section{Defining a Player}
    \item Let the hand of player $i$ be represented by a set $\{p_{i, 1}, p_{i, 2}, ..., p_{i, 12}, p_{i, 13}\}$ where $p_{i, r}$ is the number of cards of type $r$.
    \item Assume there are 4 cards of each number: $\forall_{j}\sum\limits_{i=1}^{n} p_{ij} = 4$
    \section{Defining the table}
    \item Let the table be represented as player $n+1$
    \section{Defining a Turn}
    \item Play is conducted counter-clockwise, and no player may be skipped.
    \item In a turn, consider the players $i$ and $j, i \neq j$.
    \item Let a turn ($t$) consist of the following sequence:
    \begin{enumerate}
        \item Player $i$ discards $k_t$ cards; $k_t \in [1,4], k_t \in \mathbb{N}$
        \item Player $j$ may "call BS"
        \begin{enumerate}
            \item $\forall_{i\in k_t} type(i) = \alpha_t \rightarrow$ j must take the table
            \item $\exists_{i\in k_t} type(i) \neq \alpha_t \rightarrow$ i must take the table
        \end{enumerate}
    \end{enumerate}
    \section{The Problem}
    \paragraph{} Assume you are player 1 $p_{1 r}$. We must calculate $P(p_{i \alpha_t} \geq k_t)$
    \section{Levels of Thought}
    \begin{enumerate}
        \item The table is empty and we have no knowledge of the player's cards.
        \item If a player $i$ tells the truth and player $j$ calls bluff in turn $t$, player $j$ definitely holds $k_t$ cards of type $\alpha_t$.
        \item The table is treated as another player to whom cards are given in each turn
        \item The table definitely holds the cards that you have placed into it in your previous turns, and when the table is taken by a player $i$, $i$ definitely holds those cards.
    \end{enumerate}
\end{description}
\chapter{Describing the Initial State with $\forall_{i\in\gamma} i = 0$}
\section{Assumptions} 
\paragraph*{} To begin, let us find a formula to compute the relevant probability given the table is empty and that there is no other information provided.
\paragraph*{Relevant Variables} The number of cards to distribute between $n-1$ players is $4-p_{0 \alpha_{t}}$. The number of cards player $i$ places is $k_t$, and the true amount they hold is $p_{i \alpha_t}$.
\begin{description}
    \item In general, to split $y$ items among $z$ people, There are ${y+z-1 \choose z-1}$ ways.
    \item We must count the number of ways in which $4-p_{0 \alpha_{t}}$ may be distributed between $n-1$ players. This is ${2-p_{0 \alpha_{t}} + n \choose n-2}$
    \item We must count the number of ways in which players $j$ such that $j\neq i$ can hold, which is ${1-p_{0 \alpha_{t}} - k_t + n \choose n-3}$ 
    \item We must count the number of ways in which player $i$ can hold $k_t$ cards, which is ${n-1 \choose k_t}$
    \linebreak
    \item This means that in the most naive case, the probability of player $i$ having exactly $k_t$ cards is  $\frac{{n-1 \choose k_t}\ast{1-p_{0 \alpha_{t}} - k_t + n \choose n-3}}{{2-p_{0 \alpha_{t}} + n \choose n-2}}$
    \item We must compute the probability from $k_t$ up to $4$, since a player may have more cards than what they play. $\sum\limits_{k_t}^{4-p_{0 \alpha_{t}}} \frac{{n-1 \choose k_t}\ast{1-p_{0 \alpha_{t}} - k_t + n \choose n-3}}{{2-p_{0 \alpha_{t}} + n \choose n-2}}$ 
\end{description}
\chapter{Limitations on the Player}
\paragraph{} When distributing cards, consider the limitations on the player $i, i\neq 0$. $i$ cannot hold more cards of type k than the number of cards $i$ is holding. $i$ also cannot hold more than 4 cards of type $k_t$. Finally, $i$ cannot hold more than 4-(the cards known to be held by all players $\neq i$).  
\paragraph{} For counting, $i$ must be holding at least $k_t$ cards. If $i$ is known to be holding more than $k_t$ cards of type $\alpha_t$, then $i$ cannot hold less than that number of cards.

\paragraph{} Consider the total number of ways in which cards can be distributed between all players $\neq 0$. 
\paragraph{} More than 4 cards cannot be distributed. In addition, the most cards that can be distributed (in such a way that $i$ is also relevant) must be less than $\sum{known_{\neq i}- 1$. The maximum number of cards distributed to all players cannot be more than the number of cards $i$ is holding. 
\paragraph{} At least 0 cards must be distributed. It is impossible to distribute such that $i$ is holding less than what $i$ is known to hold. Thus, we must distribute at least $known_i - 1$.
$f(r) = 0$
\paragraph{} Based on these bounds, this gives us the formula as follows.
$\frac{\sum\limits_{r=\max(known, k)}^(\min(held_i, 4-\sum{known_{\neq i}}, 4)} f(r)}{\sum\limits_{r=\max(known_i-1, 0)}^{\min(held_i,4-\sum{known_{\neq i}}, 4)} f(r)}$
\end{document}